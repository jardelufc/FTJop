\documentclass[a4paper,12pt]{scrartcl}
\usepackage{pslatex} % -- times instead of computer modern

\usepackage[colorlinks=true,linkcolor=black,citecolor=black]{hyperref}
\usepackage{booktabs}

\usepackage[latin1]{inputenc}

\newcommand{\code}[1]{{\texttt{#1}}}
\newcommand{\cmd}[1]{{\texttt{#1}}}
\newcommand{\dirent}[1]{{\texttt{#1}}}
\newcommand{\menuitem}[1]{\textsf{\textbf{#1}}}


\begin{document}



\section*{Tools for JOP Development}



For application development with JOP, the Java optimized processor,
several tools have to be installed:

%
\begin{itemize}
    \item  \href{http://java.sun.com/j2se/1.4.2/download.html}%
{Java 2 Platform 1.4, Standard Edition}  Java compiler and runtime
    \item  \href{http://www.cygwin.com/}%
{Cygwin} GNU tools for Windows. Packages cvs, gcc and make are
needed
    \item  \href{https://www.altera.com/support/software/download/altera_design/quartus_we/dnl-quartus_we.jsp}%
{Quarts II Web Edition} VHDL synthesis, place and route for Altera
FPGAs
    \item  \href{https://www.altera.com/support/software/download/programming/jam/dnl-byte_code_player.jsp}%
{Jam STAPL Byte-Code Player} FPGA configuration in batch mode
(\cmd{jbi32.exe})

\end{itemize}
%
All needed tools are freely available.

\subsection*{Java}

As we want to build Java applications we need a Java compiler. On a
standard Windows PC only the Java runtime is installed per default.
Install the Java SDK in version 1.4 (or 1.5).
Add the \dirent{bin} directory of the Java installation to
your \dirent{PATH} variable.

\subsection*{GNU Tools}

Several Unix (GNU) tools are available for Windows from cygwin. Make
sure that \code{make}, \code{gcc} and \code{cvs} are installed and
the cygwin directory is in the \dirent{PATH}.

\subsection*{Quartus}

Quartus is the VHDL compiler from Altera. The Web-edition is freely
available. However, you have to obtain a licence file (that uses the
NIC number) from Altera, which you will get per e-mail.

To enable the ByteBlaster download driver start Quartus, open the
\emph{Programmer} window with \menuitem{Tools -- Programmer}, select
\menuitem{Hardware Setup... -- Add Hardware...}, and add the
\menuitem{ByteBlaster}.

For batch FPGA configuration download \cmd{jbi32.exe} and put it in
a directory which is in the PC's PATH.

\subsection*{Optional Tools}

These tools are not absolutely necessary, but make live a little
easier.

\begin{description}
    \item[WinCvs] from \url{http://www.wincvs.org/} is a CVS client
    for Windows.
    \item[Eclipse] from \url{http://www.eclipse.org/} makes it way
    easier to write Java programs.
    \item[ModelSim] the VHDL simulator from \url{http://www.model.com/}
    is available as restricted, free version from
    \url{http://www.xilinx.com/ise/verification/mxe_details.html}.
\end{description}

\subsection*{Installation Check}

The PATH variable should contain entries to the executables of all
packages (java and javac, Cygwin bin, Quartus executables and
jbi32). Check the PATH variable at the command prompt with:
%
\begin{verbatim}
    javac
    gcc
    make
    cvs
    quartus_map
    jbi32
\end{verbatim}
%
All the executables should be found and usually report their usage.

\subsection*{Build Check}

Create a working directory and download JOP from the
\url{www.opencores.org} CVS server:
%
\begin{verbatim}
 cvs -d :pserver:anonymous@cvs.opencores.org:/cvsroot/anonymous -z9 co -P jop
\end{verbatim}
%
There is a \code{Makefile} available and when everything is set up
correct a simple
%
\begin{verbatim}
    make
\end{verbatim}
%
in the source root directory (\dirent{.../jop}) should build
everything from the sources and download a \emph{Hello World}
example.

If \url{www.opencores.org} is offline (which happens sometimes) you
can also download a source snapshot from \url{www.jopdesign.com}. A
detailed instruction for the design flow for JOP can be found in
\url{http://www.jopdesign.com/doc/build.pdf}.


\end{document}
